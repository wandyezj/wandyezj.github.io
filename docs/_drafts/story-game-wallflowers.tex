

\documentclass[letterpaper, 12pt]{article}

%
% General Notes
%

% Only use T1 fonts as these allow scaling via cm
% \usefont encoding family series shape size
% encoding
%	T1
%family
%	lmr
%	cmr
%	phv
%series
%	m - medium
%	b - bold 
%shape 
%	n - normal upright
%	it - italic
%	sl - slanted
%	sc - small caps


%
%variables
%

\newcommand{\vtitle}{Wallflowers}
\newcommand{\vauthor}{wandyezj}

\newcommand{\vreadsymbol}{$\triangleright$}

%
% Commands
% 

\newcommand{\ccard}[1] {
	\begin{tikzpicture}
		\node[draw,align=center,inner sep=0.5cm,minimum width=5.5cm] at (3,0) {#1};
	\end{tikzpicture}
}

\newcommand{\csetfontsize}[1]{
	\fontsize{#1}{#1}
	\selectfont
}

\newcommand{\csetfont}[1]{
	\usefont{T1}{#1}{x}{n}
	\selectfont
}

% read commands are to format to indicate it should be read out loud
\newcommand{\csetfontread}{
	\usefont{T1}{lmr}{m}{sl}
	\csetfontsize{0.5cm}
	%\linespread{2}
	\selectfont
	\setlength{\parskip}{2mm} % Custom spacing
	\setlength{\parindent}{0pt}
}

\newcommand{\cread}[1]{
		%\fbox{
			%\texttt{
{\csetfontread \vreadsymbol #1}
			%}
		%}
}

\newenvironment{eboxed}
{\begin{center}
		\begin{tabular}{|p{0.95\textwidth}|}
			\hline\\
		}
		{ 
			\\\\\hline
		\end{tabular} 
	\end{center}
}

\newenvironment{eread}
{
	%\begin{quote}
		
		%\begin{eboxed}
			\csetfontread
			\setlength{\parindent}{0pt}
			\vreadsymbol \linebreak
		}
		{
		%\end{eboxed}
%\end{quote}
}


%
% Packages
%



% Fonts
%\input Zallman.fd %\newcommand*\initfamily{\usefont{U}{Zallman}{xl}{n}}


%\usepackage{almendra}
%\usepackage{lmodern} 
%\usepackage[T1]{fontenc}
% Scalable Font Sizes
%\usepackage{lmodern} 
%\usepackage{type1cm}
%\usepackage[T1]{fontenc}

% Other Packages
\usepackage{geometry}

\geometry{a4paper, margin=1in}

% Drawings
\usepackage{tikz}

% Paragraphs
\usepackage{parskip}
% Framing
%\usepackage{framed}
%\usepackage{mdframed}


%\usepackage{anyfontsize}
%\usepackage{scrextend}
%\changefontsizes{18pt}




%
% Setup
%



\title{\vtitle}
\author{\vauthor}
\date{}


%
% Macro Sections
%

\newcommand{\cpagetitle}{
\begin{titlepage}
	\centering
	\vspace*{4cm}
	{
		\csetfontsize{2cm}
		\usefont{OT1}{pag}{x}{n}
		\vtitle
	}
	\vspace*{0.5cm}
	
	{\large by \vauthor}
	\vspace*{\fill}
\end{titlepage}
\pagebreak
}



%\renewcommand{\familydefault}{cmssq}
\begin{document}
{
% Global Font Control
\csetfontsize{18pt}
\csetfont{cmssq}
\fontfamily{Times New Roman}\selectfont

%
% Title Page
%

%\cpagetitle


%
% Section Start
%  Pitch
%

\section{Pitch}

\begin{quote}
TODO:

How do I pitch wallflowers in a few sentences?

Wallflowers is a game for 3 - 4 players.



%!!WARNING!!! This game can be emotionally intense. Watch yourself - let other players know if you are uncomfortable. Watch others - it's always ok to ask if people are comfortable.

\end{quote}
%\pagebreak



%
% Section Start
% Materials
%

\section{What you need to play}

\begin{itemize}
	\item 3-4 players.
	\item One player to facilitate everyone through the rules.
	\item Blank Index cards
	\item Pens
	\item (optional) Token to track whose turn it is.
\end{itemize}



%
% Section Start
% Introduction
%

%\pagebreak
\section{Pitch}

Read the following aloud:

\begin{eread}
We are playing wallflowers at a large party.

Your wallflower showed up to the party by themselves, and is now watching the party alone from the sidelines.

Initially, you know nothing about your wallflower except their alias.

What does your wallflower see, think, and feel - as they watch the party unfold?

Do they want to join the party or leave?

Don't try to answer yet!

You will discover everything about your wallflower through play.
\end{eread}


\section{Alias}


Read the following aloud:


\begin{eread}
Each player will play one wallflower.

Pick an alias for your wallflower named after a flower.

Each alias must be a single word flower name and start with a different letter.

Pick from the list below or create your own:
\end{eread}

\begin{verbatim}
Azalea
Begonia
Camellia
Dahlia
Erysimum
Foxglove
Goldenrod
Hyacinth
Indigo
Jasmine
Kudzu
Lotus
Marigold
Nerine
Orchid
Poppy
Quince
Rose
Sage
Tulip
Ulmus
Violet
Willow
Xanthium
Yarrow
Zenobia
\end{verbatim}

To help everyone keep track of who is playing each wallflower:
\linebreak
\linebreak
Have each player write down their name and their wallflower's name on a folded index card and place it in front of them.

\begin{enumerate}

\item Give each player an index card and a pen.
\item \cread{Fold your index card in half to make a tent.}
\item \cread{Write down your \textit{wallflower alias} and your name on both sides.}


Example:

\ccard{Wallflower Alias\\ \\ (Player Name)}
\ccard{Azalea\\ \\ (Ace)}

\item \cread{Place the card in front of you so everyone can see.}

\end{enumerate}


\section{Wallflower Deck}

Create the wallflower deck.

\begin{enumerate}
	\item Give each player a blank index card and have them write down their wallflower alias on one side, leaving the back blank.
		
	Example
	
	\ccard{\\Wallflower Alias\\}
	\ccard{\\Azalea\\}


	
	\item Collect these index cards.
	
\end{enumerate}


\section{Set the party}

What type of party are we playing?


\begin{enumerate}
	\item Pick an age range for our wallflowers and party attendees:
		\begin{itemize}
			\item High School
			\item College
		\end{itemize}
	\item Pick a setting:
		\begin{itemize}
			\item House party at a wealthy house
			\item Kegger in the woods
			\item Bonfire on the beach
		\end{itemize}
	

\end{enumerate}



}
\end{document}